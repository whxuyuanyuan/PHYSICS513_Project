% Use only LaTeX2e, calling the article.cls class and 12-point type.

\documentclass[12pt]{article}

% Users of the {thebibliography} environment or BibTeX should use the
% scicite.sty package, downloadable from *Science* at
% www.sciencemag.org/about/authors/prep/TeX_help/ .
% This package should properly format in-text
% reference calls and reference-list numbers.

% Use times if you have the font installed; otherwise, comment out the
% following line.
\usepackage{times}
\usepackage{url}
\usepackage{amsmath}
\usepackage{hyperref}
\usepackage{graphicx}
\usepackage{caption}
\usepackage{subcaption}
\usepackage{enumitem}
\usepackage{float}
\usepackage{physics}
\usepackage[english]{babel}
% The preamble here sets up a lot of new/revised commands and
% environments.  It's annoying, but please do *not* try to strip these
% out into a separate .sty file (which could lead to the loss of some
% information when we convert the file to other formats).  Instead, keep
% them in the preamble of your main LaTeX source file.


% The following parameters seem to provide a reasonable page setup.

\topmargin 0.0cm
\oddsidemargin 0.2cm
\textwidth 16cm 
\textheight 21cm
\footskip 1.0cm


%The next command sets up an environment for the abstract to your paper.

\newenvironment{sciabstract}{%
\begin{quote} \bf}
{\end{quote}}


% If your reference list includes text notes as well as references,
% include the following line; otherwise, comment it out.

\renewcommand\refname{References and Notes}

% The following lines set up an environment for the last note in the
% reference list, which commonly includes acknowledgments of funding,
% help, etc.  It's intended for users of BibTeX or the {thebibliography}
% environment.  Users who are hand-coding their references at the end
% using a list environment such as {enumerate} can simply add another
% item at the end, and it will be numbered automatically.

\newcounter{lastnote}
\newenvironment{scilastnote}{%
\setcounter{lastnote}{\value{enumiv}}%
\addtocounter{lastnote}{+1}%
\begin{list}%
{\arabic{lastnote}.}
{\setlength{\leftmargin}{.22in}}
{\setlength{\labelsep}{.5em}}}
{\end{list}}


% Include your paper's title here

\title{Dynamical Billiard} 


% Place the author information here.  Please hand-code the contact
% information and notecalls; do *not* use \footnote commands.  Let the
% author contact information appear immediately below the author names
% as shown.  We would also prefer that you don't change the type-size
% settings shown here.

\author
{Yuanyuan Xu,$^{1\ast}$ Zhetao Jia,$^{1}$\\
\\
\normalsize{$^{1}$Department of Physics, Duke University}\\
\\
\normalsize{\url{yx82@duke.edu}}
}

% Include the date command, but leave its argument blank.

\date{}

\usepackage[backend=bibtex]{biblatex}
\bibliography{ref}


%%%%%%%%%%%%%%%%% END OF PREAMBLE %%%%%%%%%%%%%%%%



\begin{document} 

% Double-space the manuscript.

\baselineskip24pt

% Make the title.

\maketitle 



% Place your abstract within the special {sciabstract} environment.

\begin{sciabstract}
  We have explored the dynamical behaviors of a dynamical billiard introduced by Yakov Sinai by two-dimensional computational simulation. We show the chaotic behavior of Sinai's billiard and estimate the Lyapunov exponent. We also prove the ergodic property of Sinai's billiard. 
\end{sciabstract}



% In setting up this template for *Science* papers, we've used both
% the \section* command and the \paragraph* command for topical
% divisions.  Which you use will of course depend on the type of paper
% you're writing.  Review Articles tend to have displayed headings, for
% which \section* is more appropriate; Research Articles, when they have
% formal topical divisions at all, tend to signal them with bold text
% that runs into the paragraph, for which \paragraph* is the right
% choice.  Either way, use the asterisk (*) modifier, as shown, to
% suppress numbering.

\section*{Introduction}
Formally, a billiard is a dynamical system in which a particle alternates between motion in a straight line and specular reflections from a boundary~\cite{wiki:db}. {\bf Fig.~\ref{fig:1}} \footnote{Source: George Stamatiou, Wikipedia} shows the configuration of one kind of dynamical billiards and this billiard is introduced by Yakov Sinai. Sinai's billiard arises from studying the behavior of molecules in a so-called ideal gas. In this model, we view the gas as numerous tiny balls. The simplest case is two interacting disks moving inside a 2D box. We remove a disk from the center of the box by eliminating the center of mass. And then the system is reduced to a disk bouncing inside a square boundary and outside a circular boundary.


\begin{figure}[H]
	\centering
	\includegraphics[width=0.45\linewidth]{figures/SinaiBilliard}
	\caption{Sinai's billiard}
	\label{fig:1}
\end{figure}


\section*{Computational Methods}
Since the motion of a gas molecule between two successive collisions is on a straight-line, it is convenient to record the contact points on the boundary and the direction vector after each collision. Therefore, the sequence of the contact points on the boundary is given by
\begin{equation}
	S = \{(x_i , y_i, \theta _i) : i\in \mathcal{N} \}.
\end{equation}
And the direction vector after the collision is given by
\begin{equation}
	\mathbf{\hat{v}_{i}} = (\cos{\theta_i}, \sin{\theta_i}), \quad \theta_i \in (-\pi, \pi].
\end{equation}
Then, we can get the equation of the line along the direction vector at $(x_i, y_i)$:
\begin{equation}
	\tan(\theta_i)(x - x_i) + y_i - y = 0.
\end{equation}
If the circle is centered at the origin, then the distance from the origin to the line is given by
\begin{equation}
	d = \frac{|-\tan(\theta_i) x_i + y_i |}{\tan^{2}(\theta_i) + 1}.
\end{equation}
If $d \geq R$, where $R$ is the radius of the disk, then the next contact point is on the square boundary. Otherwise, the next contact point is on the circular boundary. We can use {\tt sympy.geometry.\\intersection(line, boundary)} to determine the next contact point $(x_{i+1}, y_{i+1})$. Suppose $\mathbf{\bf{T}}$ and $\mathbf{\bf{N}}$ are the unit tangent vector and unit normal vector of the boundary at $(x_{i+1}, y_{i+1})$. Then the direction vector after collision is given by
\begin{equation}
	\mathbf{\bf{\hat{v}_{i+1}}} = (\mathbf{\hat{v}_{i}}\cdot\mathbf{\bf{T}})\mathbf{\bf{T}} - (\mathbf{\hat{v}_{i}} \cdot \mathbf{\bf{N}})\mathbf{\bf{N}}.
\end{equation} 


For two trajectories start with slightly different conditions, we can calculate the Lyapunov component by performing a linear fit on the semilog plot of $\delta(\mathbf{x})$ versus $t$. Since the motion of the ball is constrained by the boundary, we only do the curve fitting with the first few points as {\bf Fig.~\ref{fig:ly1}} shows. 


The Python source code can be found online at~\url{https://github.com/whxuyuanyuan/PHYSICS513_Project}.

\section*{Results}
\subsection*{Lyapunov Exponent\footnote{In this section, we use $R / L = 0.25$, where $R$ is the radius of the circular boundary and $L$ is the length of the side of the square boundary.}}
If two ball start with slightly different initial positions ($\| \delta_0 \| = 10^{-5}$) and we perform a linear fit on the semilog plot of $\delta(t)$ versus $t$ ({\bf Fig.~\ref{fig:ly1}}), then we can get $\lambda \approx 0.43$.
\begin{figure}[H]
	\centering
	\includegraphics[width=0.65\linewidth]{figures/Lyapunov_1}
	\caption{}
	\label{fig:ly1}
\end{figure}
{\bf Fig.~\ref{fig:ly2}} shows a small change in initial directions ($\| \delta_0) \| = 10^{-5}$) and we obtain $\lambda \approx 0.440$.  
\begin{figure}[H]
	\centering
	\includegraphics[width=0.65\linewidth]{figures/Lyapunov_2}
	\caption{}
	\label{fig:ly2}
\end{figure}

Hence, we confirm the chaotic behavior of Sinai's billiard by numerical method.

\subsection*{Transition To Chaotic Motion}
If the radius of the disk $R = 0$, then the billiard becomes an integrable system. {\bf Fig.~\ref{fig:r0_1}} shows the trajectory of the gas molecule and {\bf Fig.~\ref{fig:r0_2}} shows that the deviation of two balls with slightly different initial conditions is a linear function of time.

When $R$ is increased to $0.1$, the trajectory of the gas molecule looks like {\bf Fig.~\ref{fig:r01_1}} and the deviation of two neighboring trajectories separate exponentially after the first collision with the circular boundary as shown in {\bf Fig.~\ref{fig:r01_2}}.

Actually, the emergence of the inner boundary makes the billiard become a dispersing system. And all of the chaotic properties appear due to the dispersing boundary. 

Surprisingly, even if the system still shows quite good symmetry property(four symmetry axis) and the area of center circle is so small compared with the available region, the system turns into chaos all of a sudden. 

\begin{figure}[H] 
  \begin{subfigure}[b]{0.4\textwidth}
    \includegraphics[width=\textwidth]{figures/r0}
    \caption{}
    \label{fig:r0_1}
  \end{subfigure}
  \hfill
  \begin{subfigure}[b]{0.5\textwidth}
    \includegraphics[width=\textwidth]{figures/r0_ly}
    \caption{}
    \label{fig:r0_2}
  \end{subfigure}
  \caption{$R = 0$}
\end{figure}

\begin{figure}[H] 
  \begin{subfigure}[b]{0.4\textwidth}
    \includegraphics[width=\textwidth]{figures/r01}
    \caption{}
    \label{fig:r01_1}
  \end{subfigure}
  \hfill
  \begin{subfigure}[b]{0.5\textwidth}
    \includegraphics[width=\textwidth]{figures/r01_ly}
    \caption{}
    \label{fig:r01_2}
  \end{subfigure}
  \caption{$R = 0.1$}
\end{figure}



\subsection*{Ergodicity}

For the model of a ball bouncing around elastically inside a bounded region, the system can be classified as integrable and chaotic. 
We observe that as the bouncing time goes large enough, the trajectory tends to cover the entire plane, which is an evidence that there is no periodic motion. The following figure shows the trajectories after bouncing 1000 and 10000 times.

\begin{figure}[H] 
  \begin{subfigure}[b]{0.45\textwidth}
    \includegraphics[width=\textwidth]{figures/1000_hits.png}
    \caption{}
  \end{subfigure}
  \hfill
  \begin{subfigure}[b]{0.4\textwidth}
    \includegraphics[width=\textwidth]{figures/10000_hits.png}
    \caption{}
  \end{subfigure}
\end{figure}


To further analyze the ergodic property of the system, we sample circles of random sizes at random position inside the available region. Those circles are required to have no overlap with the center circle or any region outside the square. Then we sum over the length of trajctory lines that intersect with the circle and the sum is divided by the area of the circle. We expect this ratio to be constant as the sample size gets to infinity if the trajectory is uniformly distributed in available region as bouncing times become large enough. The following figure shows a histogram for 300 circles are sampled.

\begin{figure}[H] 
  \begin{subfigure}[b]{0.4\textwidth}
    \includegraphics[width=\textwidth]{figures/sample.jpg}
    \caption{}
  \end{subfigure}
  \hfill
  \begin{subfigure}[b]{0.55\textwidth}
    \includegraphics[width=\textwidth]{figures/histo.png}
    \caption{}
  \end{subfigure}
\end{figure}

The result shows most ratio centers around a certain value, despite of a little variance, verifies our guess of the uniform distribution of trajectories.


\subsection*{Angle Map}

Recall from the analysis of Lorentz equation, Lorentz map is introduced to show how the relation between of z-coordinate of two subsequent local maximum.\footnote{Figure 9.4.3 Strogatz} We record the angle information of every collision between the ball and  center circle. The angle $\theta$ is defined as shown in the figure in the range $[-\pi, \pi]$. The map of subsequent angles are shown in the following figure.

\begin{figure}[H] 
  \begin{subfigure}[b]{0.4\textwidth}
    \includegraphics[width=\textwidth]{figures/theta.jpg}
    \caption{}
  \end{subfigure}
  \hfill
  \begin{subfigure}[b]{0.5\textwidth}
    \includegraphics[width=\textwidth]{figures/map.jpg}
    \caption{}
  \end{subfigure}
\end{figure}

One interesting observation is that many of the mapping points concentrated inside the region close to the diagonal line, which means two subsequent angles tend to 
be close. 

\section*{Conclusions}
\begin{enumerate}
	\item Sinai's billiard has a positive Lyapunov exponent.
	\item There is a transition from integrable system to chaotic system at radius $R = 0$.
	\item The trajectory covers the entire region and the ball has equal probability to show at any point in the region after a long enough time.	\item The mapping from one hit angle to the next shows concentration in the region where the two angles are close.
\end{enumerate}


\newpage
\nocite{chernov2006chaotic, apl2014, backer2007quantum}
\printbibliography



\end{document}




















